\documentclass[10pt,a4paper]{article}
\usepackage[utf8]{inputenc}
\usepackage[russian]{babel}
\usepackage[OT1]{fontenc}
\usepackage{amsmath}
\usepackage{amsfonts}
\usepackage{amssymb}
\usepackage{mathrsfs}
\usepackage{graphicx}
\usepackage{xypic}
\usepackage{tikz}
\usepackage{hyperref}

\hypersetup{
    colorlinks=true,
    linkcolor=black,
    citecolor=black,
%    bookmarks=true,
%    filecolor=magenta,      
    urlcolor=cyan,
    pdftitle={Relations in groups and algebras related to flag polyhedral products},
    pdfauthor={F. Vylegzhanin},
%    frenchlinks=true,
%    pdfpagemode=FullScreen,
}
\usepackage{amsthm}
\usepackage{xcolor}
\usetikzlibrary{decorations.markings}
\tikzstyle arrowstyle=[scale=1]
\tikzstyle directed=[postaction={decorate,decoration={markings,
    mark=at position .65 with {\arrow[arrowstyle]{stealth}}}}]
\usepackage[left=2cm,right=2cm,top=2cm,bottom=2cm]{geometry}

\DeclareMathOperator{\id}{id}
\DeclareMathOperator{\pt}{pt}
\DeclareMathOperator{\Tor}{Tor}
\DeclareMathOperator{\Ext}{Ext}
\DeclareMathOperator{\Ker}{Ker}
\DeclareMathOperator{\Img}{Im}
\DeclareMathOperator{\cchar}{char}

\def\CC{\mathbb{C}}
\def\ZZ{\mathbb{Z}}
\def\FF{\mathbb{F}}
\def\QQ{\mathbb{Q}}
\def\RR{\mathbb{R}}
\def\NN{\mathbb{N}}
\def\R{\mathcal{R}}
\def\Z{\mathcal{Z}}
\def\A{\mathscr{A}}
\def\B{\mathscr{B}}
\def\L{\mathbf{L}}
\def\k{\mathbf{k}}
\def\OX{{\Omega X}}

\newtheorem{thm}{Теорема}[section]
\newtheorem{lmm}[thm]{Лемма}
\newtheorem{cnj}[thm]{Гипотеза}
\newtheorem{prp}[thm]{Предложение}
\newtheorem{crl}[thm]{Следствие}

\theoremstyle{definition}

\newtheorem{dfn}[thm]{Определение}
%\newtheorem{ntn}[thm]{Обозначение}
\newtheorem{rmk}[thm]{Замечание}
\newtheorem{exm}[thm]{Пример}
\author{С. Гальперин, Ж.~М. Лемэр\thanks{S. Halperin et J.~M. Lemaire}}
\title{Инертные множества в градуированных алгебрах Ли\\ ("Анатомия убийства II")\thanks{Suites inertes dans les Alg\`ebres de Lie Gradu\'ees ("Autopsie d'un meurtre II")}}
\date{}
\begin{document}
\maketitle
\section*{Введение}
Отправная точка для этой работы -- следующий вопрос, поставленный одним из нас в \cite{15}. Пусть дано односвязное пространство $X$ и элемент $\phi\in\pi_{n+1}(X);$ при каких условиях на $\phi$ вложение
$$X\to X\cup_\phi e^{n+2}$$ сюръективно на уровне гомотопических групп?

Мы исследуем этот вопрос только на уровне рациональных гомотопических групп. Напомним (см. \cite{16}), что композиция петель превращает коалгебру $H_*(\OX;\QQ)$ в алгебру Хопфа, а скобка $[\alpha,\beta]=\alpha\beta-(-1)^{|\alpha||\beta|}\beta\alpha$ задаёт на множестве её примитивных элементов структуру градуированной алгебры Ли, чьей универсальной обёртывающей является $H_*(\OX;\QQ).$ Кроме того, гомоморфизм Гуревича $\pi_*(\OX)\otimes\QQ\to H_*(\OX;\QQ)$ -- изоморфизм на эту алгебру Ли, которую, следуя Квиллену \cite{18}, мы будем обозначать $\pi(X).$

Так как все эти конструкции функториальны, сюръективность гомоморфизма
$$\pi_*(X)\otimes\QQ\to \pi_*(X\cup_\phi e^{n+2})\otimes\QQ$$ равносильна сюръективности гомоморфизма алгебр
\begin{equation}\label{eq:1}
H_*(\OX;\QQ)\to H_*(\Omega(X\cup_\phi e^{n+2});\QQ).
\end{equation}
Пусть $\langle\overline{\phi}\rangle\subset H_*(\OX;\QQ)$ -- двусторонний идеал, порождённый образом $\phi$ под действием отображения $\pi_{n+1}(X)=\pi_n(\OX)\to H_n(\OX;\QQ).$ В \cite{15} показано, что для сюръективности (\ref{eq:1})  достаточно того, чтобы проекция $H_*(\OX;\QQ)\to H_*(\OX;\QQ)/\langle \overline{\phi}\rangle$ индуцировала изоморфизм в $\Tor_i(\QQ,\QQ)$ при $i\ge 3$ и вложение в $\Tor_2(\QQ,\QQ).$ Заметим, что задействована только структура ассоциативной алгебры на $H_*(\OX;\QQ).$

Мы покажем ниже (теорема \ref{thm_1.1}), что это условие также является необходимым, отвечая на вопрос из \cite{15}. Тем самым топологический вопрос сводится к чисто алгебраическому: для каких элементов $a\in A$ градуированной ассоциативной алгебры $A$ над полем $\k$ проекция $A\to A/\langle a\rangle$ индуцирует изоморфизмы в $\Tor_i(\k,\k)$ при $i\geq 3$ и вложение в $\Tor_2(\k,\k)?$

Этот вопрос детально изучен Аником \cite{2}, и некоторые его результаты мы приводим в параграфе \ref{section:2}. По сути, Аник сформулировал другое условие на элементы ассоциативных алгебр, аналогичное условию регулярности в коммутативных алгебрах, и доказал равносильность этих условий. Он называет такие элементы \emph{сильно свободными}.

Для простоты и благозвучности мы предлагаем называть эти элементы \emph{инертными}; название мотивировано тем, что $a\in A$ инертен тогда и только тогда, когда переход $A\to A/\langle a\rangle$ меняет $\Tor_*(\k,\k)$ наименьшим возможным образом.

Итак, параграф \ref{section:1} сводит вопрос о сюръективности $$\pi_*(X)\otimes\QQ\to \pi_*(X\cup_\phi e^{n+2})\otimes\QQ$$ к вопросу, инертен ли элемент $\overline{\phi}\in H_*(\OX;\QQ).$ Но $\overline{\phi}$ -- не произвольный элемент: он принадлежит градуированной алгебре Ли $\pi(X),$ чьей универсальной обёртывающей является $H_*(\OX;\QQ).$ Естественно дать следующее определение: элемент $x\in L$ градуированной алгебры Ли над полем характеристики ноль \emph{инертен}, если он инертен в её универсальной обёртывающей $UL.$

Основные результаты параграфов \ref{section:3} и \ref{section:4}, посвящённых изучению инертности в градуированных алгебрах Ли, не переносятся на произвольные ассоциативные алгебры. Тем самым они дополняют результаты Аника. Среди этих особых свойств алгебр Ли имеем следующий критерий инертности (теорема \ref{thm_3.3}): $x\in L$ инертен тогда и только тогда, когда
\begin{enumerate}
\item[(i)] Идеал $I\subset L,$ порождённый $x,$ является свободной алгеброй Ли;
\item[(ii)] $U(L/I)$-модуль $I/[I,I],$ заданный присоединённым действием $L$ на $I,$ свободен.
\end{enumerate} 
Из него мы выводим топологический критерий (предложение \ref{prp_3.4}): $\overline{\phi}\in\pi(X)$ инертен тогда и только тогда, когда либо гомотопический слой вложения $i:X\to X\cup_\phi e^{n+2}$ -- букет не менее чем двух рациональных сфер, либо $\phi:S^{n+1}\to X$ -- рациональная гомотопическая эквивалентность.

В параграфе \ref{section:4} мы ограничиваемся алгебрами Ли, сосредоточенными в чётных размерностях, и показываем (теорема \ref{thm_4.1}), что любой элемент идеала, порождённого инертным элементов $x\in L,$ также инертен. Как приложение этого результата (пример \ref{exm_4.11}), получаем, что связная сумма $X=(S^3\times S^3)^{\# 2}$ обладает следующим свойством: рациональная категория Люстерника-Шнирельмана пространства $X\cup_\phi e^{n+2}$ всегда равна двум, каким бы ни был $\phi\in \pi_*(X).$

Среди результатов этой работы отметим следующие замечательные примеры инертных элементов. Пусть $V$ -- односвязное замкнутое $n$-мерное многообразие, и $v\in V$ -- его точка. Если рациональные когомологии $V$ не порождаются одним элементом, мы показываем (теорема \cite{thm_5.1}), что гомотопический класс маленькой $(n-1)$-мерной сферы, окружающей $v,$ инертен в $\pi(V\setminus v).$ В случае, когда $V$ формально, этот результат принадлежит Л.Аврамову \cite{3}.

Теорема \cite{thm_5.1} позволяет дать явное описание алгебр Ли связных сумм $\pi(\Omega(M\# N))$ (теорема \cite{thm_5.4}).

Эта работа -- результат сотрудничества, которое началось во время пребывания первого автора в Ницце в качестве ассоциированного профессора; теорема \cite{thm_4.1} была получена в Торонто благодаря поддержке второго автора французским CNRS и канадским CNRSG. Наконец, доказательство \cite{thm_5.1} было получено во время коллоквиума в честь Джона Мура в Принстоне; авторы рады посвятить ему эту работу, и засвидетельствовать нашу благодарность ему за работы и идеи, послужившие вдохновением для развития существенной части алгебраической топологии последних тридцати лет... и для диссертации второго автора!

В данной работе символ $\k$ всегда будет обозначать поле, и среди векторных пространств над $\k$ рассматриваются только градуированные $V=\bigoplus_{i\ge 0} V_i,$ где каждое $V_i$ конечномерно. \emph{Ряд Гильберта} для $V$ -- это формальный степенной ряд $$V(z)=\sum_{p=0}^\infty \dim(V_p) z^p,$$ и мы пишем $V(z)\ge W(z),$ когда $\dim V_p\ge \dim W_p$ для всех $p.$ Если $S$ -- клеточный комплекс конечного типа, его рядом Пуанкаре и рядом Гильберта называют ряд Гильберта его рациональных (ко)гомологий.

Все рассматриваемые алгебры градуированны и связны, аугментация и единица задаются вложениями $\k$ как компоненты степени $0.$ В параграфах \ref{section:3} и \ref{section:4} мы рассматриваем связные алгебры Ли ($L_i=0$ при $i\le 0$), дополнительно предполагая, что $\cchar \k=0.$ В параграфе \ref{section:5} мы ограничиваемся случаем $\k=\QQ.$

\tableofcontents
\section{``Вскрытие'' инертных циклов}\label{section:1}
Напомним, что для односвязных $X$ и $\phi\in\pi_{n+1}(X)$ сюръективность
$$\pi_*(X)\otimes \QQ\overset{i}{\longrightarrow} \pi_*(X\cup_\phi e^{n+2})\otimes\QQ$$ равносильна сюръективности $H_*(\OX;\QQ)\to H_*(\Omega(X\cup_\phi e^{n+2};\QQ).$ Последнее условие можно изучать с помощью моделей Адамса-Хилтона \cite{1}.

Действительно, если $\A$ -- цепная алгебра (алгебра, снабжённая дифференциалом степени -1) над произвольным полем, и $z\in \A_n$ -- цикл, мы можем рассмотреть морфизм цепных алгебр
$$\A\to \B=\A\ast T(x),~dx=z,$$ где $T(x)$ -- тензорная алгебра с образующей $x$ степени $n+1,$ а $\ast$ -- свободное произведение (копроизведение) ассоциативных алгебр. Если $\A$ -- модель Адамса-Хилтона для $\OX$ (в частности, $H\A=H_*(\OX;\k)$), а $\overline{z}\in H\A$ соответствует классу $\overline{\phi}\in H_*(\OX;\k),$ то морфизм
$$H_*(\OX;\k)\to H_*(\Omega(X\cup_\phi e^{n+2});\k)$$ отождествляется с морфизмом $H\A\overset{\alpha}{\longrightarrow} H\B,$ индуцированным вложением $\A\to \B.$

Мы собираемся показать, что $\alpha$ сюръективно тогда и только тогда, когда проекция $\pi:H\A\to H\A/\langle \overline{z}\rangle$ индуцирует изоморфизм в $\Tor_i(\k,\k),~i\geq 3,$ и вложение в $\Tor_2(\k,\k).$ Это будет доказано для произвольного цикла $z\in\A$ в произвольной цепной алгебре $\A.$

Морфизм $\alpha$ раскладывается в композицию
$$\pi:H\A\to H\A/\langle \overline{z}\rangle\quad\text{и}\quad\gamma:H\A/\langle\overline{z}\rangle\to H\B.$$
С другой стороны, определим цепную алгебру $(\overline{\B},\overline{d})$ как $\overline{\B}=H\A\ast T(x),~\overline{d}|_{H\A}=0,~\overline{d}x=\overline{z}.$ Вложение $H\A\to\overline{\B}$ индуцирует гомомофизм $$\overline{\alpha}:H\A\to H\overline{\B},$$ и мы можем сформулировать обещанный результат следующим образом:
\begin{thm}\label{thm_1.1}
В обозначениях выше, следующие условия равносильны:
\begin{enumerate}
\item[(i)] $\alpha:H\A\to H\B$ сюръективно;
\item[(ii)] $\overline{\alpha}:H\A\to H\overline{\B}$ сюръективно;
\item[(iii)] $\gamma:H\A/\langle\overline{z}\rangle \to H\B$ -- изоморфизм;
\item[(iv)] $\pi:H\A\to H\A/\langle\overline{z}\rangle$ индуцирует изоморфизм в $\Tor_3(\k,\k)$ и вложение в $\Tor_2(\k,\k);$
\item[(v)] $\pi$ индуцирует изоморфизм в $\Tor_p(\k,\k),~\forall p\ge 3,$ и вложение в $\Tor_2(\k,\k).$
\end{enumerate}
\begin{rmk}\label{rmk_1.2}
Условия (ii), (iv) и (v) относятся только к элементу $\overline{z}$ и структуре алгебры на $H\A.$
\end{rmk}
\end{thm}

Доказательство теоремы \ref{thm_1.1} опирается на следующую конструкцию. Напомним, что $n$-кратная \emph{надстройка} $s^nV$ над градуированным векторным пространством $V$ определяется как $(s^nV)_q=V_{q-n}.$ Если $A$ -- алгебра, и $r\in A_n,~n>0,$ мы определяем новую алгебру $(A\{r\},\circ)$ следующим образом: как векторное пространство,
$$A\{r\}=\k\cdot 1\oplus s^n A,$$ где $1$ имеет степень $0;$ умножение $\circ$ определяется как $$s^na\circ s^nb := s^n(arb);$$ $1$ -- единица относительно этого умножения. Ясно, что пространство $ A\{r\}_+/ (A\{r\}_+\circ A\{r\}_+)$ неразложимых элементов в $A\{r\}$ отождествляется с $s^n(A/\langle r\rangle),$ где $\langle r\rangle$ -- двусторонний идеал, порождённый $r.$ Если $\A$ -- цепная алгебра, и $z\in \A_n$ -- цикл, можно ввести на $\A\{z\}$ структуру цепной алгебры, положив $ds^n a=(-1)^n s^n(da);$ тогда
$$H(\A\{z\})=H\A\{\overline{z}\},$$ где $\overline{z}\in H(\A)$ -- класс гомологий, представленный циклом $z\in\A.$

Ключом к доказательству теоремы \ref{thm_1.1} является формулировка шестого условия, равносильного (i)-(v):
\begin{thm}\label{thm_1.3} Каждое из пяти условий теоремы \ref{thm_1.1} равносильно
\begin{enumerate}
\item[(vi)] Алгебра $H\A\{\overline{z}\}$ свободна (т.е. изоморфна тензорной алгебре).
\end{enumerate}
\end{thm}
Перед началом доказательства сделаем несколько элементарных наблюдений.

Напомним из \cite{15}, что $\B=\A\ast T(x)$ раскладывается в прямую сумму
\begin{equation}\label{eq:1.4}\B=\bigoplus_{p=0}^\infty \A\otimes (\k\cdot x\otimes \A)^{\otimes p},
\end{equation}
 где полагаем $(\bullet)^{\otimes 0}=\k.$

Напомним также, что \emph{приведённая бар-конструкция} $\mathbf{B},$ определённая в [17], -- функтор из категории связных цепных алгебр в категорию градуированных векторных пространств (мы игнорируем структуру коалгебры). Из (\ref{eq:1.4}) мы получаем
\begin{lmm}\label{lmm_1.5}
Определён изоморфизм цепных комплексов
$$\psi:\k\otimes s^{n+1}\left(\A\ast T(x)\right)\to \mathbf{B}(\A\{z\}),$$ $$s^{n+1}(a_0\otimes x\otimes a_1\otimes\dots\otimes x\otimes a_p)\mapsto s^n a_0\mid s^n a_1\mid\dots\mid s^n a_p.$$
\qed
\end{lmm}
Пусть теперь $E^r$ -- спектральная последовательность, ассоциированная с фильтрацией алгебры $\B$ степенями элемента $x$ (ср. с \cite[\S 1]{15}). Эта фильтрация отождествляется со стандартной фильтрацией на бар-конструкции, откуда следует
\begin{lmm}\label{lmm_1.6} $\psi$ индуцирует изоморфизмы
\begin{enumerate}
\item[(a)] $E_{p,q}^1\simeq \mathbf{B}_{p+1,q+n} H\A\{\overline{z}\},$
\item[(b)] $E_{p,q}^2\simeq\Tor_{p+1,q+n}^{H\A\{\overline{z}\}}(\k,\k).$\qed
\end{enumerate}
\end{lmm}
Немедленное следствие:
\begin{lmm}\label{lmm_1.7}
$\mathrm{(vi})\Leftrightarrow E_{p,*}^2=0,~p\ge 1.$\qed
\end{lmm}
\begin{proof}[Доказательство теорем \ref{thm_1.1} и \ref{thm_1.3}]
Будем доказывать следующие импликации:
$$\xymatrix{
\mathrm{(ii)}\ar@{=>}[r] & \mathrm{(i)}\ar@{=>}[d] & \mathrm{(iii)}\ar@{=>}[l] & \\
     & \mathrm{(vi)}\ar@{=>}[r]\ar@{=>}[lu] & \mathrm{(iv)}\ar@{=>}[u]\ar@{<=>}[r] & \mathrm{(v)}}$$ Все, кроме $\mathrm{(i)}\Rightarrow \mathrm{(vi)},$ немедленно следуют из рассуждений из \cite[\S 2]{15} и леммы \ref{lmm_1.7}, если заметить, что $\Img\overline{\alpha}=E_{0,*}^2$ и $\Img\alpha =E_{0,*}^\infty.$
     
Чтобы установить $\mathrm{(i)}\Rightarrow \mathrm{(vi)},$ выберем градуированное подпространство $U\subset \A,~d|_U=0,$ так, чтобы композиция $U\to H\A\overset{\alpha}{\longrightarrow} H\B$ была изоморфизмом, и рассмотрим гомоморфизм цепных алгебр
     $$\chi:(T(s^nU),\circ)\to\A\{z\},$$ построенный по вложению $s^nU\to s^n\A.$ Достаточно показать, что $\chi$ индуцирует изоморфизм в гомологиях. Рассмотрим коммутативную диаграмму
$$\xymatrix{
\mathbf{B}T(s^nU)\ar[r]^{\mathbf{B}\chi} & \mathbf{B}\A\{z\}\\
& \k\oplus s^{n+1}\B\ar[u]_\psi^\simeq\\
\k\oplus s^{n+1} U\ar[uu]^\sim\ar[ur]^\sim\ar[r] & \k\oplus s^{n+1}\A.\ar[u]_\subseteq}$$ 
Гомоморфизм слева индуцирует изоморфизм в гомологиях, т.к. $T(s^nU)$ свободная; диагональный гомоморфизм -- по построению $U.$ Получаем, что $\mathbf{B}\chi$ индуцирует изоморфизм в гомологиях, поэтому то же верно для $\chi$ по \emph{теореме Мура о сравнении} \cite{17}.
\end{proof}
\section{Инертные элементы и инертные множества в алгебрах}\label{section:2}
Пусть $A$ -- алгебра. Напомним, что каждому элементу $r\in A_n,~n>0,$ сопоставляется двусторонний идеал $\langle r\rangle \subset A$ и проекция $\pi:A\to A/\langle r\rangle.$
\begin{dfn}\label{dfn_2.1}
Элемент $r\in A$ \emph{инертен}, если выполнено одно из двух условий, эквивалентных за счёт теоремы \ref{thm_1.3}:
\begin{enumerate}
\item[(a)] Алгебра $A\{r\}$ свободна;
\item[(b)] Морфизм $\Tor_p^{\pi}(\k,\k)$ инъективен при $p=2$ и биективен при $p=3.$
\end{enumerate}
\end{dfn}
\begin{rmk} Из (a) немедленно следует, что централизатор инертного элемента -- это подалгебра, порождённая им.
\end{rmk}

Рассмотрим теперь конечный или счётный набор элементов $(r_i),~r_i\in A_{n_i},~n_i>0,$ и обозначим как $\langle r_1,\dots, r_s\rangle\subset A$ двусторонний идеал, порождённый $r_1,\dots,r_s.$
\begin{dfn}
Набор $(r_i)\subset A$ \emph{инертен}, если $r_1\in A$ инертен, и для всех $i$ образ элемента $r_{i+1}$ инертен в $A/\langle r_1,\dots, r_i\rangle.$
\end{dfn}

Пусть $I\subset A$ -- двусторонний идеал, порождённый конечным или счётным набором $(r_i),~r_i\in A_{n_i},~n_i>0.$ Обозначив $I^0=A,~I^{k+1}=I^k\cdot I,$ получаем \emph{$I$-адическую} убывающую фильтрацию на $A.$ Обозначим как $\mathrm{Gr}\,A$ биградуированную алгебру, ассоциированную с этой фильтрацией:
$$\mathrm{Gr}_{p,q}A=(I^p/I^{p+1})_{p+q}.$$ В частности, $\mathrm{Gr}_{0,*}A=A/I,~\mathrm{Gr}_{1,*}A=I/I^2.$
Пусть $\overline{Q}I=I/(A_+\cdot I+I\cdot A_+)$ -- пространство неразложимых элементов идеала $I.$ Любой базис ``дополнения'' $A_+\cdot I+I\cdot A_+$ до $I$ является минимальной системой образующих идеала $I,$ и наоборот. Сечение $\lambda:\overline{Q}I\to I/I^2$ индуцирует гомоморфизм биградуированных алгебр:
$$\chi:A/I\ast T(\overline{Q}I)\to \mathrm{Gr}\,A;$$ его ограничение на $A/I$ -- тождественное отображение.

Классическое рассуждение (см., например, \cite[(2.2)]{2}) показывает, что $\chi$ сюръективен для любого идеала $I.$ Инъективность же, как заметил Аник, равносильна инертности минимальной системы образующих для $I.$ Для удобства читателя приведём основные результаты \cite[\S 2]{2}.  Пусть $X$ -- градуированное векторное пространство с базисом $(x_i),~|x_i|=n_i+1,$ $A$ -- алгебра, и $A\ast T(X)$ снабжена дифференциалом $d,$ для которого $d|_{A}=0,~dx_i=r_i.$ Разные описания инертных множеств, данные Аником, можно объединить в следующую теорему:
\begin{thm}\label{thm_2.4}
Следующие условия равносильны:
\begin{enumerate}
\item[(i)] Набор $(r_i)$ инертен в $A;$
\item[(ii)] $(r_i)$ -- базис в $\overline{Q}I,$ и $\chi$ -- изоморфизм;
\item[(iii)] $(r_i)$ -- базис в $\overline{Q}I,$ и ряды Гильберта для $A,A/I,\overline{Q}I$ связаны как $\frac{1}{A(z)}=\frac{1}{A/I(z)}-\overline{Q}I(z);$
\item[(iv)]$(r_i)$ -- базис в $\overline{Q}I,$ и проекция $\pi:A\to A/I$ индуцирует изоморфизм в $\Tor_3(\k,\k)$ и вложение в $\Tor_2(\k,\k);$
\item[(v)] Гомоморфизм алгебр $\alpha:A\to H(A\ast T(X))$ сюръективен;
\item[(vi)] $\alpha$ индуцирует изоморфизм $A/I\overset{\simeq}{\longrightarrow} H(A\ast T(X)).$
\end{enumerate}\qed
\end{thm}
Как немедленное следствие теорем \ref{thm_1.1} и \ref{thm_2.4} получаем
\begin{crl}\label{crl_2.5}
Если $S=T\oplus U$ -- подпространство в $A_+,$ то объединение базисов $T$ и $U$ инертно в $A$ тогда и только тогда, когда базис $T$ инертен в $A,$ а образ базиса $U$ инертен в $A/I_T$ (здесь $I_T$ -- двусторонний идеал, порождённый $T$).\qed
\end{crl}
Условие инертности относится не сколько к самому набору, сколько к порождённому им идеалу, при условии, что оно порождает этот идеал минимальным образом. В частности, любая перестановка инертного множества инертна. Мы приходим к следующему определению:
\begin{dfn}\label{dfn_2.6}
Двусторонний идеал $I\subset A$ \emph{инертен}, если выполнено одно из двух эквивалентных условий:
\begin{enumerate}
\item[(i)] $I$ порождён инертным множеством;
\item[(ii)] Проекция $\pi:A\to A/I$ индуцирует вложение в $\Tor_2(\k,\k)$ и биекцию в $\Tor_3(\k,\k).$
\end{enumerate}
\end{dfn}
Пусть теперь $(F_pA)_{p\in\ZZ}$ -- произвольная фильтрация на алгебре $A,$ конечной в каждой размерности, и $E_{*,*}^0A=\mathrm{Gr}\,A$ -- ассоциированная с ней биградуированная алгебра. Следующее предложение доказывается тем же образом, что и \cite[теорема 3.2]{2}:
\begin{prp}\label{prp_2.9}
Если $(\overline{r}_i)\in E_{*,*}^0A$ -- инертный набор биоднородных элементов, и $(r_i)$ -- прообразы $(\overline{r}_i),$ то набор $(r_i)$ инертен.
\end{prp}
\section{Инертные элементы и инертные множества в алгебрах Ли}\label{section:3}
С этого момента мы предполагаем, что $\cchar \k=0.$

Напомним, что свободная алгебра Ли $\L(X)$ над градуированным векторным пространством $X$ -- это подалгебра Ли в $T(X),$ порождённая $X;$ её универсальная обёртывающая $U\L(X)$ отождествляется с $T(X).$ Функтор универсальной обёртывающей коммутирует с копроизведениями, откуда $UL\ast T(X)=U(L\ast \L(X))$ для любой алгебры Ли $L.$

Теорема Пуанкаре-Биркгофа-Витта, верная и в градуированном случае (см. \cite{16}, \cite[аппендикс B]{18}), отождествляет $L$ с множеством примитивных элементов алгебры Хопфа $UL.$ Та же теорема позволяет показать, что если $J\subset UL$ -- двусторонний идеал, порождённый идеалом Ли $I\subset L,$ то
$$I=L\cap J,\quad[L,I]=L\cap(J\cdot UL_+ + UL_+\cdot J),\quad U(L/I)=UL/J.$$
Пространство неразложимых элементов идеала Ли $I,$ которое мы обозначим $\overline{Q}I=L/[L,I],$ отождествляется с $\overline{Q}J=J/(J\cdot UL_++UL_+\cdot J),$ поэтому обозначение $\overline{Q}$ не приводит к путанице. Более общо, если $I^{(0)}=L,~I^{(k)}=[I^{(k-1}),I]$ обозначает убывающую центральную фильтрацию идеала $I,$ то $$I^{(k)}=L\cap J^k,$$ и получаем естественный изоморфизм биградуированных алгебр
$$U(\mathrm{Gr}\,L) = \mathrm{Gr}\, UL,$$ где $\mathrm{Gr}\, UL$ ассоциирована с $J$-адической фильтрацией на $UL.$

Выбор сечения $\overline{Q}I=I/[L,I]\to I/[I,I]$ позволяет определить гомоморфизм
$$\chi_L:(L/I)\ast \L(\overline{Q}I)\to \mathrm{Gr}\,L,$$ причём
$$U\chi_L:(UL/J)\ast T(\overline{Q}J)\to \mathrm{Gr}\, UL$$ совпадает с гомоморфизмом $\chi,$ определённым в параграфе \ref{section:2}.

Эти замечания позволяют перенести на алгебры Ли результаты из параграфа \ref{section:2}, дав следующее естественное определение:
\begin{dfn}\label{dfn_3.1}
Элемент (набор) \emph{инертен} в алгебре Ли $L,$ если он инертен в алгебре $UL;$ идеал Ли $I\subset L$ инертен, если $J\subset UL$ инертен.
\end{dfn}
Заметим, что минимальная система образующих идеала Ли $I$ инертна тогда и только тогда, когда $I$ инертен.

За счёт сделанных замечаний теорема \ref{thm_2.4} переносится на случай алгебр Ли очевидным образом:
\begin{prp}\label{prp_3.2} Следующие условия эквивалентны:
\begin{enumerate}
\item[(i)] Набор $(r_i)$ инертен в $L;$
\item[(ii)] $(r_i)$ -- базис в $\overline{Q}I,$ и $\chi_L$ -- изоморфизм;
\item[(iii)]$(r_i)$ -- базис в $\overline{Q}I,$ и факторизация $L\to L/I$ индуцирует изоморфизм в $\Tor_3^{U\bullet}(\k,\k)$ и вложение в $\Tor_2^{U\bullet}(\k,\k);$
\item[(iv)] Отображение $L\to H(L\ast \L(Y),d),$ где $Y$ -- векторное пространство с базисом $(y_i),~|y_i|=|r_i|+1,~d|_L=0,~dy_i=r_i,$ сюръективно.
\end{enumerate}
\qed\end{prp}
Для идеалов в алгебрах Ли можно дать полезный критерий инертности, не имеющий аналогов в случае произвольных ассоциативных алгебр.

Пусть $I\subset L$ -- идеал. Действие $L$ на $I$ индуцирует действие $L/I$ на $QI=I/[I,I];$ оно наделяет $QI$ естественной структурой $U(L/I)$-модуля, для которого
$$\overline{Q}I=I/[L,I]=QI/((U(L/I)_+\cdot QI)=\k\otimes_{U(L/I)}QI$$ является пространством неразложимых элементов.
\begin{thm}\label{thm_3.3}
Идеал $I\subset L$ инертен тогда и только тогда, когда выполнены оба условия:
\begin{enumerate}
\item[(i)] $I$ -- свободная алгебра Ли: $I\simeq \L(QI);$
\item[(ii)] $QI$ -- свободный $U(L/I)$-модуль.
\end{enumerate}
\end{thm}
Эта теорема имеет следующее топологическое приложение Пусть $X$ -- односвязное клеточное пространство конечного типа, и $\phi:S^{n+1}\to X$ -- непрерывное отображение. Обозначим $F=\mathrm{hofib}(X\overset{i}{\longrightarrow} X\cup_\phi e^{n+2}).$
\begin{prp}\label{prp_3.4}
За исключением тривиального случая, когда $\phi$ -- рациональная гомотопическая эквивалентность, следующие условия эквивалентны:
\begin{enumerate}
\item[(i)] $\pi_*(X)\otimes\QQ \overset{i_{\#}}{\longrightarrow} \pi_*(X\cup_\phi e^{n+2})\otimes\QQ$ сюръективно;
\item[(ii)] элемент $\overline{\phi}\in\pi(X)$ инертен;
\item[(iii)] $F$ имеет рациональный гомотопический тип букета $\ge 2$ сфер.
\end{enumerate}
\end{prp}
\begin{proof}
Эквивалентность (i) и (ii) показана в теореме \ref{thm_1.1}. Если (i) верно, то ядро $i_\#$ совпадает с $\pi_*(F)\otimes \QQ;$ следовательно, $\pi(F)$ -- ядро сюръекции $\pi(X)\to \pi(X\cup_\phi e^{n+2}).$ Условие (i) из \ref{thm_3.3} показывает, что $F$ -- рациональный букет сфер. Так как $X\cup_\phi e^{n+2}$ не рационально стягиваемо, из условия (ii) теоремы \ref{thm_3.3} вытекает $\dim \widetilde{H}_*(F;\QQ)\ge 2.$

Наоборот: если $F$ -- букет не менее чем двух сфер, то $\pi(F)$ -- свободная алгебра Ли от $\ge 2$ образующих. Значит, её центр тривиален и тем самым совпадает с образом связывающего гомоморфизма $\pi_*(X\cup_\phi e^{n+2})\otimes\QQ\to \pi_{*-1}(F)\otimes \QQ,$ откуда $i_\#$ сюръективно.
\end{proof}
\begin{rmk}\label{rmk_3.5}
В \cite{9} Феликс и Томас показали, что $\widetilde{H}_*(F;\QQ)$ -- свободный $U\pi(X\cup_\phi e^{n+2})$-модуль, даже если $\phi$ не инертен.
\end{rmk}
\begin{proof}[Доказательство теоремы \ref{thm_3.3}]
Точной последовательности алгебр Ли $I\to L\to L/I$ соответствует спектральная последовательность
$$E_{p,q}^2 = \Tor_p^{U(L/I)}(\Tor_q^{UL}(U(L/I),\k),\k)\Rightarrow \Tor_{p+q}^{UL}(\k,\k).$$
Изоморфизм замены колец $$\Tor_q^{UL}(U(L/I),\k)\overset{\simeq}{\longrightarrow} \Tor_q^{UI}(\k,\k)$$ снабжает $\Tor_q^{UI}(\k,\k)$ структурой $U(L/I)$-модуля. При $q=1$ верно $\Tor_1^{UI}(\k,\k)=QI,$ и эта структура совпадает с определением выше (см. \cite[3.12]{7}).

Пусть $I$ инертен. Гомоморфизм $$\Tor^{UL}(\k,\k)\to \Tor^{U(L/I)}(\k,\k)$$ отождествляется с граничным гомоморфизмом $E_{*,*}^\infty\to E_{*,0}^2,$ который тем самым инъективен при $p+q=2$ и биективен при $p+q\ge 3.$ Получаем $E_{1,1}^2=E_{1,1}^\infty =0,$ то есть
$$\Tor_1^{U(L/I)}(\Tor_1^{UI}(\k,\k),\k)=0.$$

{\color{red} тут я не то чтобы проследил}

Значит, множество $QI=\Tor_1^{UI}(\k,\k)$ -- свободный $U(L/I)$-модуль, и $\Tor_p^{U(L/I)}(QI,\k)=E_{p,1}^2=0$ при $p\geq 1.$ Значит, второй лист спектральной последовательности сосредоточен в строке $E_{*,0}^2$ и, возможно, в элементе $E_{0,1}^2:$ действительно, из углового гомоморфизма
$$E_{0,2}^2=\Tor_0^{U(L/I)}(\Tor_2^{UI}(\k,\k),\k)=0$$
и, тем самым, $\Tor_2^{UI}(\k,\k)=0$ по ``лемме Накаямы'', т.к. $U(L/I)$ связная. Это означает, что $UI$ (следовательно, и $I$) свободна. Доказательство в обратную сторону проделывается по той же схеме.
\end{proof}
\begin{exm}\label{exm_3.6} Элемент $b\in\L(a,b)$ инертен, т.к. идеал $(b)$ -- свободная алгебра Ли с образующими $(b_n)_{n=0}^\infty,$ $b_0=b,~b_{n+1}=[a,b_n].$
\end{exm}
\begin{exm}\label{exm_3.7}
Пусть $L=\L(a,x)/([x,x]),$ где $|x|$ нечётна. Ясно, что $L=\L(a)\ast \k\cdot x,$ где $\k\cdot x$ -- одномерная коммутативная алгебра Ли. По второму пункту \ref{prp_3.2}, $a\in L$ инертен. Хотя $L/(a)=\k\cdot x,$ и $U(L/(a))=\Lambda[x],$ подалгебра Ли $(a)\subset L$ свободно порождена элементами $a$ и $[x,a]:$ имеем расширение
$$0\to \L(a,[x,a])\to L\to \k\cdot x\to 0,$$ откуда ясно, что $L$ содержит свободную подалгебру Ли коразмерности 1.
\end{exm}

Инертный элемент ассоциативной алгебры может не быть инертным в её подалгебрах: например, $ab$ инертен в $T(a,b),$ но соотношение $ab\cdot a^2b=aba\cdot ab$ показывает, что $ab$ не инертен в подалгебре, порождённой $ab,~a^2b$ и $aba.$ В алгебрах Ли такого не может произойти:
\begin{thm}\label{thm_3.8}
Пусть $E\subset L$ -- подалгебра Ли, и $(r_i)\subset E.$ Если $(r_i)$ инертен в $L,$ то $(r_i)$ инертен в $E.$
\end{thm}
\begin{lmm}\label{lmm_3.9}
Пусть $E\subset L$ -- подалгебра Ли, $I\subset L$ -- инертный идеал, и $I\subset E.$ Тогда $I\subset E$ инертен, и любой набор элементов, минимально порождающих $I$ как идеал в $L,$ является инертным множеством в $E.$
\end{lmm}
\begin{proof}
По теореме \ref{thm_3.3}, $I$ -- свободная алгебра, а $QI$ -- свободный $U(L/I)$-модуль. Так как $U(L/I)$ -- свободный $U(E/I)$-модуль (по теореме Пуанкаре-Биркгофа-Витта), $QI$ -- свободный $U(E/I)$-модуль. Значит, $I$ инертен в $E$ по теореме \ref{thm_3.3}.

{\color{red} лажа?}

Минимальный набор образующих $I$ как идеала в $L$ можно дополнить до минимального набор образующих $I$ как идеала в $E.$ Подмножество инертного множества инертно.
\end{proof}

\begin{proof}[Доказательство теоремы \ref{thm_3.8}] Пусть набор элементов $(r_i)\subset E$ инертен в $L.$ Рассмотрим убывающую последовательность алгебр Ли $E^{(k)}:$
$$E^{(0)}=L,\quad E^{(k+1)}=E+I^{(k)},$$ где $I^{(k)}$ -- идеал в $E^{(k)},$ порождённый множеством $(r_i).$ По лемме \ref{lmm_3.9}, множество $(r_i)$ инертно в каждой из $E^{(k)}.$ Пусть $I$ -- идеал, порождённый $(r_i)$ в $E.$ Вложения
$$E\hookrightarrow E^{(k)},\quad I\to I^{(k)}$$ -- изоморфизмы в степенях $\le k.$ Значит, то же верно для гомоморфизмов алгебр
$$U(E/I)\to U(E^{(k)}/I^{(k)}).$$

Из инертности $I^{(k)}\subset E^{(k)}$ получаем, что при $p+q<k$ верно: гомоморфизм
$$\Tor_{p,q}^{UE}(\k,\k)\to \Tor_{p,q}^{U(E/I)}(\k,\k)$$ инъективен при $p=2$ и биективен при $p=3.$ Так как $k$ произвольно, $I$ -- инертный идеал. Так как $(r_i)$ -- минимальный набор порождающих своего идеала в $L,$ то же верно и для идеала в $E.$ 
\end{proof}
{\color{red} не дописал}

\section{Инертность элементов инертных идеалов}\label{section:4}
{\color{red} не дописал}

\section{Гомотопические группы ``проколотых'' многообразий}\label{section:5}
Мы завершаем эту работу результатом, показывающим замечательные примеры инертных гомотопических классов. Назовём \emph{моногенной} ассоциативную градуированную алгебру, порождённую одним элементом. Если $V$ -- многообразие и $v\in V,$ будем обозначать ``проколотое'' многообразие $V\setminus v$ как $V^\circ.$

\begin{thm}\label{thm_5.1}
Пусть $V$ -- многообразие (или, более общо, $\QQ$-комплекс Пуанкаре \cite{6}), компактное, без границы, односвязное, $n$-мерное, и пусть $v\in V.$ Тогда: если $H^*(V;\QQ)$ не моногенна, то вложение $V^\circ\hookrightarrow V$ индуцирует сюръекцию
$$\pi(V^\circ)\to\pi(V).$$
\end{thm}
Другими словами, класс вложения маленькой $(n-1)$-сферы, окружающей $v,$ инертен в $\pi(V^\circ);$ или ``приклеивание $n$-мерной клетки в минимальном клеточном разбиении $V$ инертно''.

В частном случае, когда $V$ -- формальное пространство, эта теорема принадлежит Л.Аврамову \cite[теорема 7.5]{3}: интересно заметить, что Аврамов вывел этот результат из свойств горенштейновых локальных колец с помощью ``топологически -- локально-алгебраического словаря'', обнаруженного Дж.-Е. Роосом и дополнявшегося в течение трёх лет Аврамовым и Феликсом, среди прочих (см. \cite{3}, \cite{4}). Теорема \ref{thm_5.1} также уточняет результат Сташеффа \cite{19}, по которому рациональный гомотопический тип $V$ однозначно восстанавливается по рациональному типу $V^\circ$ и алгебре $H^*(V;\QQ).$ Наша теорема явно выражает алгебру $\pi(V)$ через $\pi(V^\circ),$ если $H^*(V;\QQ)$ не моногенна.

Если $H^*(V;\QQ)$ моногенна и имеет вид $P(x)/x^{m+1},$ клеточное рассуждение показывает, что $H^*(V^\circ;\QQ)$ тоже имеет вид $P(x)/x^m.$ Пространства $V,V^\circ$ тем самым формальны и их алгебры Ли имеют вид
$$\pi(V)=\QQ\cdot \alpha\oplus\QQ\cdot\beta;\quad \pi(V^\circ)=\QQ\cdot\alpha\oplus\QQ\cdot\gamma,$$ где $|\alpha|=|x|-1,$ $|\beta|=(m+1)|x|-2,$ $|\gamma|=m|x|-2.$ Морфизм $\pi(V^\circ)\to \pi(V)$ отображает $\alpha\mapsto\alpha,~\gamma\to 0.$ 

Наконец, если $V$ -- рациональная сфера, то $V^\circ$ рационально стягиваемо.

{\color{red} не дописал}
%Перед доказательством теоремы \ref{thm_5.1} приведём приложение к связным суммам: пусть $V,W$ -- (замкнутые, односвязные) $n$-мерные многообразия, не являющиеся рациональными сферами. Пусть $V\# W$ -- их связная сумма, и $\alpha\in \pi_{n-2}(V\# V)$ представляет элемент, получаемый прокалыванием 


\bibliographystyle{abbrv}
\bibliography{minibibliography} %Can add other bibliographies by comma separating (no spaces)
\end{document}