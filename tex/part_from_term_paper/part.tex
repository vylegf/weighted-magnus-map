\documentclass[10pt,a4paper]{article}
\usepackage[utf8]{inputenc}
\usepackage[russian]{babel}
\usepackage[OT1]{fontenc}
\usepackage{amsmath}
\usepackage{amsfonts}
\usepackage{amssymb}
\usepackage{graphicx}
\usepackage{xypic}
\usepackage{tikz}
\usepackage{hyperref}
\usepackage{amsthm}
\usepackage{xcolor}
\usetikzlibrary{decorations.markings}
\tikzstyle arrowstyle=[scale=1]
\tikzstyle directed=[postaction={decorate,decoration={markings,
    mark=at position .65 with {\arrow[arrowstyle]{stealth}}}}]
\usepackage[left=2cm,right=2cm,top=2cm,bottom=2cm]{geometry}
\author{Федор Вылегжанин}
\title{О возможной связи между соотношениями в группах и алгебрах, связанных с флаговыми полиэдральными произведениями}
\date{}

\newcommand{\Bigast}{\mathop{\scalebox{1.5}{\raisebox{-0.2ex}{$\ast$}}}}%


%\DeclareMathOperator{\rk}{rk}
\DeclareMathOperator{\id}{id}
\DeclareMathOperator{\pt}{pt}
\DeclareMathOperator{\supp}{supp}
\DeclareMathOperator{\rank}{rank}
\DeclareMathOperator{\cone}{cone}
\DeclareMathOperator{\Tor}{Tor}
\DeclareMathOperator{\Ext}{Ext}
\DeclareMathOperator{\Ker}{Ker}
\DeclareMathOperator{\Img}{Im}
\DeclareMathOperator{\pr}{proper}
\DeclareMathOperator{\ab}{ab}
\DeclareMathOperator{\sk}{sk}

\def\CC{\mathbb{C}}
\def\ZZ{\mathbb{Z}}
\def\FF{\mathbb{F}}
\def\QQ{\mathbb{Q}}
\def\RR{\mathbb{R}}
\def\NN{\mathbb{N}}
\def\Nm{\NN_{\geq 0}^m}
\def\RC{\mathrm{RC}}
\def\RA{\mathrm{RA}}
\def\R{\mathcal{R}}
\def\Z{\mathcal{Z}}
\def\K{\mathcal{K}}
\def\Kt{\widetilde{\K}}
\def\pk{\pi_\K}
\def\pkt{\pi_\Kt}
\def\ZK{\Z_\K}
\def\RK{\R_\K}
\def\RKt{\R_{\Kt}}
\def\RCK{\RC_\K}
\def\RCKt{\RC_{\Kt}}
\def\OZK{\Omega\ZK}
\def\DJ{\CC\mathrm{P}^\infty}
\def\DJK{(\DJ)^\K}
\def\ODJK{\Omega\DJK}


\newtheorem{thm}{Теорема}[section]
\newtheorem{lmm}[thm]{Лемма}
\newtheorem{cnj}[thm]{Гипотеза}
\newtheorem{prp}[thm]{Предложение}
\newtheorem{crl}[thm]{Следствие}

\theoremstyle{definition}

\newtheorem{dfn}[thm]{Определение}
%\newtheorem{ntn}[thm]{Обозначение}
\newtheorem{rmk}[thm]{Замечание}
\newtheorem{exm}[thm]{Пример}
\begin{document}
\maketitle



\subsection{Группы, связанные с полиэдральными произведениями}
%Определим ``алгебраические полиэдральные произведения'' по аналогии с полиэдральными произведениями как копределы соответствующих диаграмм, тип которых -- малая категория граней симплициального комплекса $\K$. Умножение в группах и алгебрах задействует только два элемента, поэтому результат зависит только от 1-остова симплициального комплекса. Мы приводим только явные конструкции.
\begin{dfn}
Пусть $G_1,\dots,G_m$ -- группы, $\K$ -- симплициальный комплекс. \emph{Граф-произведением групп} называется группа
$$(\underline{G})^\K:=\Bigast_{i=1}^m G_i /\left(g_ig_j=g_jg_i,~g_i\in G_i,~g_j\in G_j,~\{i,j\}\in\K\right).$$
\end{dfn}
Ясно, что граф-произведение зависит только от 1-остова $\K.$ 
\begin{dfn} Взяв в предыдущем определении %$G_i=\ZZ,$ получим \emph{прямоугольную группу Артина}
%$$\RA_\K:=\langle g_1,\dots, g_m\mid g_ig_j=g_jg_i,~\{i,j\}\in\K\rangle;$$
%взяв
$G_i=\ZZ_2,$ получим \emph{прямоугольную группу Кокстера}
$$\RC_\K:=\langle g_1,\dots, g_m\mid g_i^2=\id,~i=1\dots m;~g_ig_j=g_jg_i,~\{i,j\}\in\K\rangle.$$
\end{dfn}

\begin{prp}[\cite{prv,pv}]
Пусть $G_1,\dots,G_m$ -- топологические группы, $\K$ -- симплициальный комплекс. Тогда есть каноническое гомотопическое расслоение
$$(E\underline{G},\underline{G})^\K\to(B\underline{G})^\K\to\prod_{i=1}^mBG_i.$$\qed
\end{prp}
\begin{crl}[\cite{pv}, теорема 3.2]
Пусть $G_1,\dots,G_m$ -- дискретные группы. Тогда
\begin{enumerate}
\item $\pi_1((B\underline{G})^\K)\simeq (\underline{G})^\K;$
\item $\pi_k((B\underline{G})^\K)\simeq\pi_k((E\underline{G},\underline{G})^\K),~k\geq 2;$ 
\item $(E\underline{G},\underline{G})^\K$ и $(B\underline{G})^\K$ асферичны тогда и только тогда, когда $\K$ флаговый;
\item Имеем точную последовательность фундаментальных групп
$$1\to\pi_1((E\underline{G},\underline{G})^\K)\to (\underline{G})^\K\overset{\pi}{\longrightarrow}\prod_{i=1}^m G_i\to 0.$$
\end{enumerate}\qed
\end{crl}
Если все $G_i$ абелевы, то в последней точной последовательности $\pi$ -- это абелианизация. Поэтому в этих случаях 
$$\pi_1((E\underline{G},\underline{G})^\K)\simeq ((\underline{G})^\K)'.$$
Тем самым с помощью полиэдральных произведений можно изучать коммутанты граф-произведений абелевых групп, или, в общем случае, \emph{декартовы подгруппы} $\Ker((\underline{G})^\K\to\prod_{i=1}^m G_i)$.

Рассмотрение частного случая $G_i=\ZZ_2$ даёт
\begin{crl}
Пусть $\K$ -- флаговый симплициальный комплекс. Тогда
\begin{enumerate}
\item $\pi_1((\RR\mathrm{P}^\infty)^\K)\simeq \RCK;$
\item $\RK$ и $(\RR\mathrm{P}^\infty)^\K$ асферичны;
\item $\pi_1(\RK)\simeq\RC'_\K.$
\end{enumerate}
\end{crl}
\begin{proof}
Достаточно заметить, что $B\ZZ_2=\RR\mathrm{P}^\infty,$ а пара $(D^1,S^0)$ гомотопически эквивалентна паре $(S^\infty,S^0)=(E\ZZ_2,\ZZ_2).$
\end{proof}

\subsection{Алгебры, связанные с полиэдральными произведениями}

\begin{dfn}
Градуированной \emph{алгеброй} Ли будем называть обычную алгебру Ли, обладающую градуировкой (выполняется обычное тождество Якоби и обычная антикоммутативность $[y,x]=-[x,y]$).

Градуированной \emph{супералгеброй} Ли назовём градуированную абелеву группу с градуированно-антикоммутативной скобкой и градуированным тождеством Якоби:
$$[y,x]=-(-1)^{|x||y|}[x,y];\quad (-1)^{|x||z|}[x,[y,z]]+(-1)^{|x||y|}[y,[z,x]]+(-1)^{|y||z|}[z,[x,y]]=0.$$

Например, если $X$ -- топологическое пространство, то $\pi_*(\Omega X)\otimes\QQ$ -- градуированная супералгебра Ли (вместе с произведением Самельсона, т.е. со смещённым произведением Уайтхеда).

Свободная алгебра Ли обозначается как $FL(u_1,\dots,u_m);$ свободная супералгебра Ли -- как $FSL(u_1,\dots,u_m);$ свободная ассоциативная алгебра (т.е. тензорная алгебра) -- как $T(u_1,\dots,u_m).$
\end{dfn}
\begin{dfn}
Над любым кольцом с единицей можно определить граф-произведение алгебр и супералгебр Ли:
$$L_\K := FL(u_1,\dots, u_m)/\left([u_i,u_i]=0,~i=1\dots m;~[u_i,u_j]=0,~\{i,j\}\in\K\right);$$
$$SL_\K := FSL(u_1,\dots, u_m)/\left([u_i,u_i]=0,~i=1\dots m;~[u_i,u_j]=0,~\{i,j\}\in\K\right);$$
\end{dfn}

\begin{prp}[\cite{ToricTopology}, Предложение 8.4.1]
Имеем точную последовательность градуированных супералгебр Ли
$$0\to\pi_*(\OZK)\otimes\QQ\to\pi_*(\DJK)\otimes \QQ\to CL(u_1,\dots,u_m)\to 0$$

и алгебр Понтрягина
$$0\to H_*(\OZK;k)\to H_*(\DJK;k)\to \Lambda_k[u_1,\dots,u_m]\to 0;$$
$k$ -- произвольное коммутативное кольцо с единицей.\qed\end{prp}

\begin{prp}[\cite{pr}] Если $\K$ флаговый, то $\pi_*(\DJK)\otimes\QQ\simeq SL_\K.$\qed
\end{prp}
За счёт теоремы Милнора-Мура из этого предложения вытекает
\begin{crl} Если $\K$ флаговый, $\FF$ -- поле характеристики ноль, то
$$H_*(\DJK;\FF)\simeq T(u_1,\dots, u_m)/\left(u_i^2=0,~i=1\dots m;~u_iu_j+u_ju_i=0,~\{i,j\}\in\K\right).$$\qed
\end{crl}
\subsection{Образующие и соотношения (флаговый случай)}

\begin{prp}[\cite{pv_artin}, теорема 5.2]
Пусть $\K$ флаговый. Тогда $\pi_1((E\underline{G},\underline{G})^\K)=
\Ker((\underline{G})^\K\to\prod_{i=1}^m G_i)$ имеет минимальный набор образующих, состоящий из всех вложенных коммутаторов вида
$$(g_{k_1},(g_{k_2},\dots,(g_{k_{l-2}},(g_j,g_i))\dots)),$$
где $g_k\in G_k\setminus\{\id\},~k_1<\dots<k_{l-2}<j>i,~k_s\neq i,~\forall s;~i$ -- наименьшая вершина в некоторой компоненте связности подкомплекса $\K_{\{k_1,\dots,k_{l-2},j,i\}},$ не содержащей $j.$\qed
\end{prp}
Частный случай:
\begin{prp}[\cite{pv}, теорема 4.5]
Пусть $\K$ флаговый. Тогда $\pi_1(\RK)=\RC'_\K$ имеет минимальный набор образующих, состоящий из всех вложенных коммутаторов вида
$$(g_{k_1},(g_{k_2},\dots,(g_{k_{l-2}},(g_j,g_i))\dots)),$$
где $g_k$ -- $k$-ая образующая $\RCK,$ $k_1<\dots<k_{l-2}<j>i,~k_s\neq i,~\forall s;~i$ -- наименьшая вершина в некоторой компоненте связности подкомплекса $\K_{\{k_1,\dots,k_{l-2},j,i\}},$ не содержащей $j.$\qed
\end{prp}

\begin{prp}[\cite{pv}, теорема 4.3]
Пусть $\K$ флаговый. $\pi_1((E\underline{G},\underline{G})^\K)=
\Ker((\underline{G})^\K\to\prod_{i=1}^m G_i)$ свободна тогда и только тогда, когда $\K^1$ -- хордовый граф.\qed
\end{prp}

\begin{prp}[\cite{gptw}, теорема 4.3]
Пусть $\mathcal{K}$ флаговый, $\FF$ -- поле. Тогда $H_*(\OZK;\mathbb{F})$ имеет минимальный набор мультипликативных образующих, состоящий из всех вложенных коммутаторов вида
$$[u_{k_1},[u_{k_2},\dots,[u_{k_{l-2}},[u_j,u_i]]\dots]],$$
где $k_1<\dots<k_{l-2}<j>i,~k_s\neq i,~\forall s;~i$ -- наименьшая вершина в некоторой компоненте связности подкомплекса $\K_{\{k_1,\dots,k_{l-2},j,i\}},$ не содержащей $j.$\qed
\end{prp}
\begin{prp}[\cite{gptw}]
Пусть $\K$ флаговый, $\FF$ -- поле. $H_*(\OZK;\mathbb{F})$ свободна тогда и только тогда, когда $\K^1$ -- хордовый граф.\qed
\end{prp}
\subsection{Мультиградуировка; ряды Пуанкаре}
\begin{dfn}
Пусть $X$ -- топологическое пространство. Его числа Бетти и эйлерова характеристика определяются как
$$b_i(X):=\dim H_i(X),~\chi(X):=\sum_{i} (-1)^i b_i(X).$$
Определим также \emph{приведённые} числа Бетти и эйлерову характеристику:
$$\widetilde{b}_i(X):=\dim\widetilde{H}_i(X),~\widetilde{\chi}(X):=\sum_{i}(-1)^i\,\widetilde{b}_i(X).$$
Очевидно, $\widetilde{b}_i(X)=b_i(X)$ при $i\neq \{0,-1\};$ если $X\neq\varnothing,$ то $$\widetilde{b}_0(X)=b_0(X)-1,\quad\widetilde{b}_{-1}(X)=b_{-1}(X)=0.$$
Отдельно рассматривается случай $X=\varnothing:$ имеем $$\widetilde{b}_0(\varnothing)=b_0(\varnothing)=0,\quad\widetilde{b}_{-1}(\varnothing)=0,~b_{-1}(\varnothing)=1.$$
В любом случае $\widetilde{\chi}(X)=\chi(X)-1.$
\end{dfn}

Под мультиградуировкой всегда будем понимать градуировку ассоциативной алгебры элементами полугруппы $\mathbb{N}_{\geq 0}^m.$ Базисные векторы обозначим как $e_1,\dots,e_m;$ степень однородного элемента $a$ -- как $|a|.$ Если $J\subset[m],$ вместо степени $\sum_{j\in J}e_j\in\Nm$ будем писать просто ``степень $J$''. Часто, хотя и не всегда, алгебра будет порождена $m$ образующими, и $i$-ая образующая имеет степень $e_i.$
\begin{dfn}
Пусть $V$ -- мультиградуированное векторное пространство. Его \emph{рядом Пуанкаре} будем называть формальный степенной ряд от $m$ переменных $\lambda=(\lambda_1,\dots,\lambda_m)$
$$F(V;\lambda):=\sum_{\alpha\in \mathbb{N}_{\geq 0}^m} \dim(V_\alpha)\cdot \lambda^\alpha,\quad\lambda^\alpha:=\prod_{i=1}^m \lambda_i^{\alpha_ie_i}.$$
\end{dfn}
Аналогично соглашению выше, если $J\subset[m],$ вместо $\lambda^{\sum_{j\in J}e_j}=\prod_{j\in J}\lambda_j$ будем писать просто $\lambda^J.$

Следующее почти очевидное предложение несколько раз используется в разделе 6.3.
\begin{prp}
Ряд Пуанкаре обладает следующими свойствами:
\begin{enumerate}
\item $F(V_1\oplus V_2;\lambda)=F(V_1;\lambda)+F(V_2;\lambda);$
\item $F(V_1\otimes V_2;\lambda)=F(V_1;\lambda)\cdot F(V_2;\lambda);$
\item Если $0\to A_1\to A_2\to A_3\to 0$ -- точная последовательность алгебр Хопфа, то $$F(A_2;\lambda)=F(A_1;\lambda)\cdot F(A_3;\lambda).$$
\end{enumerate}
\end{prp}
\begin{proof}
Очевидно, если $\{v_i\}_{i\in I}$ -- базис векторного пространства $V,$ то $F(V;\lambda)=\sum_{i\in I}\lambda^{|v_i|}.$
\begin{enumerate}
\item Базис прямой суммы -- это объединение базисов.
\item Базис тензорного произведения -- это попарные тензорные произведения базисных векторов.
\item Как векторное пространство, расширение одной алгебры Хопфа с помощью другой -- это тензорное произведение.
\end{enumerate}
\end{proof}


Напомним, что имеют место следующие точные последовательности (групп и алгебр Ли):
$$1\to \pi_1(\RK)\hookrightarrow (\ZZ_2)^\K \overset{\ab}{\longrightarrow} (\ZZ_2)^m\to 0,$$
$$0\to \pi_*(\OZK)\otimes\FF\hookrightarrow SL^\K \overset{\ab}{\longrightarrow} CL(u_1,\dots,u_m)\to 0,$$
где
$$(\ZZ_2)^\K=F(g_1,\dots,g_m)/\left(g_i^2=\id,~i=1\dots m;~(g_i,g_j)=\id,~\{i,j\}\in\K\right),$$
$$SL^\K=FSL(u_1,\dots,u_m)/\left([u_i,u_i]=0,~i=1\dots m;~[u_i,u_j]=0,~\{i,j\}\in \K\right).$$
\subsection{Отображение Магнуса и нижний центральный ряд}
Пусть $G$ -- произвольная группа. Обозначим $(g,h):=g^{-1}h^{-1}gh$ для $g,h\in G$ и $(A,B):=\{(a,b):~a\in A,~b\in B\}$ для $A,B\subset G.$
\begin{dfn}
\emph{Нижний центральный ряд} группы определяется индуктивно: $$\gamma_1(G):=G;~\gamma_{n+1}(G):=(G,\gamma_n(G)).$$
\end{dfn}
\begin{prp}[\cite{magnus}] Градуированная группа
$$L(G):=\bigoplus_{k=1}^\infty \gamma_k(G)/\gamma_{k+1}(G)$$ обладает естественной структурой алгебры Ли относительно группового коммутатора. Её называют \emph{присоединённой алгеброй Ли} группы $G.$
\end{prp}
Определим теперь вложение Магнуса. Пусть $X$ -- фиксированное множество, $F(X)$ -- свободная группа с базисом $X.$ Рассмотрим кольцо $A=\ZZ\langle\!\langle X\rangle\!\rangle$ некоммутативных ассоциативных формальных степенных рядов от переменных $\alpha_x,~\forall x\in X$ со стандартной $\NN_{\geq 0}$-градуировкой. Это свободная ассоциативная алгебра с базисом $X;$ в частности, это универсальная обёртывающая свободной алгебры Ли $FL(X).$ Элементы $FL(X)$ в $A$ называются \emph{лиевыми элементами}.

Группу обратимых элементов $A$ будем обозначать как $A^\times.$
\begin{dfn}
\emph{Вложение Магнуса} -- это гомоморфизм $\mu:F(X)\to A^\times,$ заданный на образующих как $x\mapsto 1+\alpha_x.$ 
\end{dfn}
Каждому степенному ряду $\sigma\in A^\times\setminus\{1\}$ сопоставим его \emph{девиацию} -- первое нетривиальное однородное слагаемое, отличное от 1. Запишем формулой:
$$\delta(\sigma):=\sigma_i,\text{ где }i=\min\{j\in\NN:~\sigma_j\neq 0\}.$$
По определению также положим $\delta(1):=0.$
\begin{prp}[\cite{magnus}]
\begin{enumerate}
\item $\mu$ инъективно;
\item Если $w\in F(X),~w\neq\id,$ то $\delta(\mu(w))$ -- лиев элемент;
\item Сопоставление $w\mapsto \delta(\mu(w))$ индуцирует изоморфизм алгебр Ли $L(F(X))\to FL(X).$
\end{enumerate}
\end{prp}

Вложение Магнуса можно использовать на практике для поиска соотношений в присоединённых алгебрах Ли. А именно: пусть $G=\langle x_1,\dots, x_N\mid r_1,\dots, r_M\rangle$ -- любое копредставление. Ясно, что сюръекция $\pi:F(X)\to G$ индукцирует $L(\pi):FL(X)\to L(G),$ то есть некоторое копредставление алгебры Ли $L(G).$

Каждому соотношению $r_i\in F(X)$ соответствует  однородный элемент в $[r_i]\in FL(X)$ (более точно: если $r_i\in \gamma_k(F(X))\setminus \gamma_{k+1}(F(X)),$ то $[r_i]\in FL_k(X)$). Этот элемент лежит в ядре $L(\pi),$ то есть является соотношением в $L(G);$ его можно вычислить с помощью вложения Магнуса. В общем случае могут быть и другие соотношения.
\subsection{Взвешенное отображение Магнуса}
Рассмотрим обобщение классической конструкции, описанной выше. Пусть $G=\langle x_1,\dots, x_N\mid r_1,\dots, r_M\rangle$ -- фиксированное копредставление, и выбраны элементы $d_1,\dots, d_N\in \Nm$ -- ``степени образующих''.

Градуируем алгебру $A=\ZZ\langle\!\langle a_1,\dots, a_N\rangle\!\rangle$ как $|a_i|:=d_i$ и рассмотрим вложение Магнуса $F(X)\overset{\mu}{\longrightarrow} A^\times.$ Для слова $w\in F(X)$ формальный степенной ряд $\mu(w)\in A^\times$ теперь может иметь сильно больше однородных компонент, являющихся лиевыми элементами; возможно, есть способ формализовать это в форме некого гомоморфизма алгебр Ли, но пока это неясно.
\subsection{Формулировка гипотезы}
Вспомним, что образующие $\RC'_\K$ имеют вид
$$\Gamma_{i\in J}=(g_{k_1},(g_{k_2},\dots,(g_{k_{s-2}},(g_j,g_i))\dots))$$ для некоторых наборов $J=\{g_{k_1},\dots,g_{k_{s-2}},g_j,g_i\}\subset [m].$ Припишем образующей $\Gamma_{i\in J}$ степень $d_{i\in J}:=\sum_{j\in J}e_j\in \Nm.$ Это даст $\Nm$-градуировку на $\ZZ\langle\!\langle a_1,\dots, a_N\rangle\!\rangle.$

Также мы можем рассмотреть коммутаторную подалгебру $[L^\K,L^\K]$ в граф-алгебре Ли $L_\K;$ ``с точностью до знаков'', она аналогична коммутаторной подалгебре Ли $[SL^\K,SL^\K]=\pi_*(\OZK)\otimes\FF.$ Поэтому $[L^\K,L^\K]$ имеет базис такого же вида, что и $[SL^\K,SL^\K]:$ доказательство проходит по той же схеме. Этот базис имеет ту же мощность, что и набор образующих $\RC'_\K;$ градуируем его тем же образом. Свободная ассоциативная алгебра $A=\ZZ\langle\!\langle a_1,\dots, a_n\rangle\!\rangle$ -- это универсальная обёртывающая свободной алгебры Ли $FL(a_1,\dots,a_N),$ которую можно отобразить в $L^\K,$ задав тем самым $L^\K$ образующими и соотношениями. Это значит, что у лиевых элементов алгебры $A$ корректно определены образы в $L^\K.$

Из доказательства теоремы 3.6 ясно, что можно выбрать соотношения между образующими $\RC'_\K$ так, что каждое соотношение ``относится'' к какому-то из полных подкомплексов $\K_J,~J\subset[m].$
\begin{cnj} Пусть соотношение $R\in F(X)$ в группе $\RC'_\K$ относится к полному подкомплексу $\K_J.$ Тогда градуированная компонента степенного ряда $\mu(R)\in A^\times,$ имеющая степень $\sum_{j\in J} e_j,$ -- лиев элемент, который является соотношением в $L^\K.$
\end{cnj}
\subsection{Примеры}
\subsubsection{Граница квадрата}
%\boxed{
		\begin{tikzpicture}[scale=1, inner sep=2mm]
		\coordinate (1) at (0,0);
		\coordinate (2) at (1,0);
		\coordinate (4) at (0,1);
		\coordinate (3) at (1,1);
		
		% Draw the edges
		\draw (1) -- (2) -- (3) -- (4) -- (1);
		
		% Draw all of the vertices
		\foreach \i in {1, ..., 4} {\fill (\i) circle (2.5pt);}
		
		% Label the vertices
		\draw (1) node[left] {$1$};
		\draw (2) node[right] {$2$};
		\draw (3) node[right] {$3$};
		\draw (4) node[left] {$4$};
		\end{tikzpicture}
		
У коммутанта группы $$\RCK=\langle g_1,g_2,g_3,g_4\mid g_i^2=\id;~(g_1,g_2)=(g_2,g_3)=(g_3,g_4)=(g_4,g_1)=\id\rangle$$ две образующие $$x_1=(g_3,g_1),~x_2=(g_4,g_2)$$ и единственное соотношение $$R_{1234}=x_1^{-1}x_2^{-1}x_1x_2\in F(x_1,x_2).$$
Имеем $$\deg x_1 = e_3+e_1=(1,0,1,0),~\deg x_2=e_4+e_2=(0,1,0,1).$$
Слову $R_{1234}$ соответствует формальный степенной ряд
\begin{equation*}
r_{1234}=(1+a_1)^{-1}(1+a_2)^{-1}(1+a_1)(1+a_2)=1+\underbrace{(a_1a_2-a_2a_1)}_{\deg = (1,1,1,1)}+\underbrace{(a_1a_2a_1-a_1a_1a_2)}_{\deg=(2,1,2,1)}+\underbrace{(a_2a_2a_1-a_2a_1a_2)}_{\deg=(1,2,1,2)}+\dots\in A^\times.
\end{equation*}

Взяв градуированную компоненту степени $e_1+e_2+e_3+e_4,$ получаем соотношение $a_1a_2-a_2a_1,$ то есть $[a_1,a_2]=0.$ Это действительно соотношение между $a_1=[u_3,u_1]$ и $a_2=[u_4,u_2]$ -- образующими коммутаторной подалгебры в алгебре Ли
$$L^\K=\langle u_1,u_2,u_3,u_4\mid [u_i,u_i]=\id;~[u_1,u_2]=[u_2,u_3]=[u_3,u_4]=[u_4,u_1]=0\rangle.$$
То же соотношение верно в $SL^\K.$

\subsubsection{Комплекс $\K_3$}
		\begin{tikzpicture}[scale=1, inner sep=2mm]
		\coordinate (3) at (0,-0.2);
		\coordinate (5) at (-0.6,-1.1);
		\coordinate (6) at (0.6,-1.1);
		\coordinate (2) at (-1,0);
		\coordinate (4) at (1,0);
		\coordinate (1) at (0,1);
%		\coordinate (a) at (-1.5, 0.5);
		
		% Draw the edges
		\draw (1) -- (2) -- (5) -- (6) --(4) -- (3)  -- (1);
		\draw (2) -- (3);
		\draw (1) -- (4);
%		\draw (1)  to [out=90,in=-135] (4);
%		\draw[dashed] (1) to [out=0,in=-140] (5);
%		\draw[dashed] (1) -- (6);
%		\draw[dashed] (2)--(6);
		
		% Draw all of the vertices
		\foreach \i in {1, ..., 6} {\fill (\i) circle (2.5pt);}
		
		% Label the vertices
		\draw (1) node[left] {$1$};
		\draw (2) node[left] {$2$};
		\draw (3) node[below] {$3$};
		\draw (4) node[right] {$4$};
		\draw (5) node[left] {$5$};
		\draw (6) node[right] {$6$};
%		\draw (a) node[left] {(a)};
		\end{tikzpicture}
		Это третий из десяти минимальных симплициальных комплексов, таких что в $H^*(\ZK)$ есть нетривиальные произведения Масси.
		
		Образующие в $\RC'_\K$ (для наглядности вместо некоторых $x_i$ пишем $y_i$ и $z_i$):
$$x_1=(g_5,g_1),~\deg x_1 = (1,0,0,0,1,0);\quad\qquad\qquad y_1=(g_2,(g_6,g_1)),~\deg y_1 = (1,1,0,0,0,1);$$
$$x_2=(g_6,g_1),~\deg x_2 = (1,0,0,0,0,1);\quad\qquad\qquad y_2=(g_3,(g_5,g_1)),~\deg y_2 = (1,0,1,0,1,0);$$
$$x_3=(g_4,g_2),~\deg x_3 = (0,1,0,1,0,0);\quad\qquad\qquad y_3=(g_3,(g_6,g_1)),~\deg y_3 = (1,0,1,0,0,1);$$
$$x_4=(g_6,g_2),~\deg x_4 = (0,1,0,0,0,1);\quad\qquad\qquad y_4=(g_4,(g_5,g_1)),~\deg y_4 = (1,0,0,1,1,0);$$
$$x_5=(g_5,g_3),~\deg x_5 = (0,0,1,0,1,0);\quad\qquad\qquad y_5=(g_5,(g_6,g_1)),~\deg y_5 = (1,0,0,0,1,1);$$
$$x_6=(g_6,g_3),~\deg x_6 = (0,0,1,0,0,1);\quad\qquad\qquad y_6=(g_3,(g_6,g_2)),~\deg y_6 = (0,1,1,0,0,1);$$
$$x_7=(g_5,g_4),~\deg x_7 = (0,0,0,1,1,0);\quad\qquad\qquad y_7=(g_2,(g_5,g_4)),~\deg y_7 = (0,1,0,1,1,0);$$
$$z_1=(g_2,(g_3,(g_6,g_1))),~\deg z_1 =(1,1,1,0,0,1);\quad y_8=(g_4,(g_6,g_2)),~\deg y_8 = (0,1,0,1,0,1);$$
$$z_2=(g_3,(g_4,(g_5,g_1))),~\deg z_2 =(1,0,1,1,1,0);\quad y_9=(g_4,(g_5,g_3)),~\deg y_9 = (0,0,1,1,1,0);$$
$$z_3=(g_3,(g_5,(g_6,g_1))),~\deg z_3 =(1,0,1,0,1,1);\quad y_{10}=(g_5,(g_6,g_3)),~\deg y_{10} = (0,0,1,0,1,1).$$

	Им соответствуют образующие в $L^\K:~a_1,\dots,c_3,$ которые получаются заменой $g_i$ на $u_i,$ круглых скобок на квадратные.

Соотношения в фундаментальной группе (подкомплексов с нетривиальными $\Pi_1$ три: две границы пятиугольника и весь комплекс целиком):
$$R_{12456}=x_4x_2y_1^{-1}x_1^{-1}x_7y_7^{-1}x_3^{-1}x_1y_4^{-1}x_3y_1x_2^{-1}x_3^{-1}y_8x_4^{-1}x_2y_4x_1^{-1}x_7^{-1}y_5x_2^{-1}x_7x_4y_8^{-1}x_3y_7x_7^{-1}x_4x_2y_5^{-1}x_1x_2^{-1},$$
$$R_{23456}=x_4y_6^{-1}x_5^{-1}x_7y_7^{-1}x_3x_5^{-1}y_9^{-1}x_3y_6x_4^{-1}x_6^{-1}x_4x_3^{-1}y_8x_4^{-1}x_6y_9x_5^{-1}x_7^{-1}y_{10}x_6^{-1}x_7x_4y_8^{-1}x_3y_7x_7^{-1}x_4^{-1}x_6y_{10}^{-1}x_5,$$
$$R_{123456}=x_4y_6^{-1}x_2y_1^{-1}z_1y_3^{-1}y_2x_1^{-1}x_5{-1}x_7y_7^{-1}x_3^{-1}x_5y_9^{-1}x_1y_2^{-1}z_2y_4^{-1}x_3y_3z_1^{-1}y_1\cdot$$ $$\cdot x_2^{-1}y_6x_4^{-1}x_6^{-1}x_4x_3^{-1}y_8x_4^{-1}x_6x_2y_3^{-1}y_4z_2^{-1}y_2x_1^{-1}y_9x_5^{-1}x_7^{-1}x_5y_5z_3^{-1}y_3x_2^{-1}x_5^{-1}\cdot$$ $$\cdot y_{10}x_6^{-1}x_7x_4y_8^{-1}x_3y_7x_7^{-1}x_4^{-1}x_6y_{10}^{-1}x_5x_2y_3^{-1}z_3y_5^{-1}x_1y_2^{-1}y_3x_2^{-1}.$$
Надо вычислить градуированные компоненты соответствующих степенных рядов:
\begin{equation*}
r_{12456}=1+\underbrace{([a_1,b_8]-[a_2,b_7]-[a_3,b_5]-[a_4,b_4]+[a_7,b_1]
)}_{\deg=(1,1,0,1,1,1)}+\dots,
\end{equation*}
\begin{equation*}
r_{23456}=1+\underbrace{([a_3,b_{10}]+[a_4,b_9]-[a_5,b_8]+[a_6,b_7]-[a_7,b_6])}_{\deg = (0,1,1,1,1,1)}+\dots,
\end{equation*}
\begin{equation*}
r_{123456}=1+\dots+\underbrace{(-[a_1,[a_3,a_6]]-[a_2,[a_3,a_5]]
 +[a_3,c_3]+[a_4,c_2]-[a_7,c_1]
 -[b_2,b_8]+[b_3,b_7]+[b_1,b_9]-[b_4,b_6])}_{\deg=(1,1,1,1,1,1)}+\dots
 \end{equation*}
 
Вычисления (с помощью пакета SuperLie для Wolfram Mathematica) показывают, что выражения в скобках действительно являются соотношениями в $L^\K.$

В $SL^\K$ верны похожие тождества, где у некоторых слагаемых другие знаки:
\begin{equation*}
[a_1,b_8]+[a_2,b_7]+[a_3,b_5]+[a_4,b_4]-[a_7,b_1]=0,
\end{equation*}
\begin{equation*}
[a_3,b_{10}]+[a_4,b_9]-[a_5,b_8]+[a_6,b_7]+[a_7,b_6]=0,
\end{equation*}
\begin{equation*}
-[a_1,[a_3,a_6]]+[a_2,[a_3,a_5]]
 +[a_3,c_3]+[a_4,c_2]+[a_7,c_1]
 -[b_2,b_8]+[b_3,b_7]+[b_1,b_9]+[b_4,b_6]=0.
 \end{equation*}
 
\bibliographystyle{abbrv}
\bibliography{minibibliography}
%\begin{thebibliography}{9}

\end{document}